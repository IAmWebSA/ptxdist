% ----------------------------------------------------------------------------
\chapter{Maintaining Userlands} 		\label{chap:userlands}
% ----------------------------------------------------------------------------

\begin{flushright}
\emph{
Use the source, Luke!\\
}
\textsc{\small Obi-Wan Kenobi}
\end{flushright}

The idea for PTXdist came up when the people at Pengutronix had to
compile the userlands for customer projects over and over again. Nearly
no customer had the same requirements: one needed network tools like
\texttt{ifconfig}, servers like \texttt{thttpd} or \texttt{dhcpd},
others needed tools for accessing displays, embedded web browsers or
other Human Machine Interface stuff. 

The systems most time were so different that the way of stripping down a
mainstream distribution did not work well: you simply cannot strip down
a Debian system small enough to fit into a 2~MB RAM and 2~MB Flash
system without destroying the whole idea behind the distribution. The
other way is to build a Userland from scratch: Linux distributions like
LFS follow this path, but users have to do every configuration and
compilation step one after the other, which is a time consuming and
errorprone way. 

What we basically needed when we started PTXdist was: 
\begin{itemize}
\item A configuration frontend which lets the embedded developer just
      select which components he wants to have in his Userland. The
      configuration should be totally independend from the rest of the
      system. 
\item The system should be small at this time: no need to download
      several tens of megabytes from the net, even before selecting what
      would go into the image.
\item A set of rules, easily readable by every reasonably silled Unix
      develper, which reads the configuration and defines how to 
      build a certain software packet, taking into account what the 
      user has selected during configuration. 
\item A framework which makes it easy for a developer to add new packets
      to the system when he needs something which is not yet integrated. 
\item The whole thing should be Open Source like the Linux kernel and
      the Userland software itself, as our customers wanted a completely
      transparent, modificable and debuggable software framework, be it 
      the bootloader, the kernel or, finally, the Userland tools. 
\item Last but not least, the whole system should be platform
      independend, to build userland for whatever architecture the 
      user works with. 
\end{itemize}

When we looked around for existing solutions there was none that
fulfilled all these requirements: some where large beasts (like the
uClinux distribution), some had an interesting approach (like the
emDebian Project) but used the wrong tools (CML2 for configuration,
which is basically a dead project today), others had a great
infrastructure for the rules (like buildroot, part of the uClibc
project) but no configuration frontend usable by normal end users. So we
finally had a look at all the available projects, collected the best
ideas and started a new Open Source project: PTXdist. 

Over the last two years, PTXdist grew with the projects which have been
developed for Pengutronix' customers. PTXdist is still no distribution:
distributions are meant to just-work-out-of-the-box, which is simply not
possible for the multifaceted situations we have in embedded projects.
The idea behind PTXdist is more something like: 

\begin{important}
{\bf PTXdist is "`Executable Documentation"'. Write down what you 
did once to make something work, in a well defined manner, and
implicitly don't forget it until you need it the next time.}
\end{important}

When a Userland for a certain customer board is developed, the PTXdist
framework is used to document all the necessary steps to build the
binaries from the original sources. PTXdist defines which configuration
switches have to be selected, which tweaks have to be applied to buggy
upstream source trees and which are the commands which have to be
started in which order to build a working tree. 

At the end of a development project, the customer normally has a
standard configuration integrated in PTXdist which lets him build his
own Userland with \texttt{make~world}, he is able to work with the
framework and switch on and off things he wants to change himself. Some
customers have enough skills to send patches in case some randomly
switched on config button doesn't show the expected results, others use
Pengutronix' support services to integrate new packets into PTXdist or
fix problems which come up with configuration constellations which have
beed never tested before (with over 600 config entries there are $2^600$
possible combinations -- it would take some time to test them all). 

And, of course, we managed to build a community around PTXdist, which
consists mainly of a mailing list on the PTXdist server. People outside
Pengutronix started to work with the system for their own project, which
is very important to make PTXdist more robust and test it in
constellations which have not yet been realized here.  

So when you start your next Embedded Linux project with PTXdist don't
expect it to be something "`finished"' -- it is as finished or
unfinished as the Linux kernel itself, which is just a collection of
software which has proven to be great in one or the other situation, and
which has been looked over by several people. The more people use
PTXdist for their projects, the more robust and the more feature rich it
becomes -- just like with the kernel. The power of Linux and PTXdist
come from it's Open Source nature and from people working together to
form an industry quality software framework -- you just have to join the
team!

